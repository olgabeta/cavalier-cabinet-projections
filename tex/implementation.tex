\chapter{Υλοποίηση κώδικα}

Το βασικότερο μέλημα της παρούσας εργασίας είναι η δημιουργία ενός προγράμματος (\textattachfile[color = 0 0 1]{program.cpp}{\textlatin{program.cpp}}) διαδραστικής περιστροφής μοναδιαίου κύβου πέριξ των αξόνων $x$, $y$ και $z$, καθώς και η απεικόνιση των προβολών προοπτικής, \textlatin{Cavalier} και \textlatin{Cabinet} του κύβου αυτού. Το πρόγραμμα έχει δημιουργηθεί σε κώδικα \textlatin{C++}, με τη χρήση του λογισμικού \textlatin{OpenGL}. Για την ανάπτυξη του κώδικα συμβουλευτήκαμε τους Δρακόπουλο (2022), Γκάνια (2016) και \textlatin{Thormählen} (2021). Ακολουθεί η εξήγηση του προγράμματος, το οποίο δύναται να χωριστεί σε έξι βασικά στάδια. \par

Αρχικό βήμα της υλοποίησης είναι η δήλωση των απαραίτητων βιβλιοθηκών, τόσο αυτών της \textlatin{OpenGL} όσο και μαθηματικών συναρτήσεων και αλφαριθμητικών, καθώς και η δήλωση των διαστάσεων του παραθύρου. Στο δεύτερο στάδιο δηλώνουμε τους τρεις τύπους απεικονιζόμενων προβολών - ισομετρική προβολή, προβολή \textlatin{Cavalier} και προβολή \textlatin{Cabinet} - θέτοντας την ισομετρική ως την προκαθορισμένη προβολή εμφανιζόμενη κατά τη δημιουργία παραθύρου απεικόνισης. Στη συνέχεια, σχεδιάζουμε τον μοναδιαίο κύβο με τη χρήση της εντολής \textlatin{GL\_LINE\_LOOP} της \textlatin{OpenGL}, καθώς και τους άξονες $z$, $x$ και $y$ με τη χρήση της εντολής \textlatin{GL\_LINES}. Ο σχεδιασμός των δύο τελευταίων βασίζεται στη διαδικασία δημιουργίας του πρώτου άξονα και του βέλους κατεύθυνσής του, το οποίο σχεδιάζεται με τις εντολές \textlatin{GL\_TRIANGLES} και \textlatin{GL\_POLYGON}. \par

Το τρίτο στάδιο αποτελείται από τη συνάρτηση απεικόνισης \emph{\textlatin{handleRender()}}. Η συνάρτηση αυτή ορίζει το μητρώο ταυτότητας (\textlatin{identity matrix}), υπολογίζοντας τις γωνίες και τα μητρώα της ισομετρικής προβολής και των πλάγιων προβολών \textlatin{Cavalier} και \textlatin{Cabinet}. Επιπροσθέτως, σχεδιάζει τους τρεις άξονες, καλώντας τις αντίστοιχες συναρτήσεις και χρωματίζοντας μπλε τον άξονα $z$, πράσινο τον άξονα $y$ και κόκκινο τον άξονα $x$. Τέλος, καλεί τη συνάρτηση σχεδιασμού του μοναδιαίου κύβου. \par

Κατά το τέταρτο βήμα, ορίζουμε τη συνάρτηση ανακαθορισμού μεγέθους του παραθύρου \emph{\textlatin{handleReshape()}}, η οποία ορίζει το επίπεδο και τα $x$, $y$, $w$, $h$ με την χρήση των εντολών \textlatin{glViewport} και \textlatin{glOrtho}. Αναλυτικότερα, τα $x$ και $y$ καθορίζουν την κάτω αριστερά γωνία της θύρας προβολής σε \textlatin{pixel}, ενώ το πλάτος $w$ και το ύψος $h$ αντιπροσωπεύουν το ορθογώνιο προβολής, σχεδιάζοντας ξανά το παράθυρο σύμφωνα με τις αλλαγές. Έπειτα ακολουθεί το πέμπτο στάδιο, δηλαδή η συνάρτηση αλληλεπίδρασης με το πληκτρολόγιο \emph{\textlatin{handleKeydown()}}. Η συνάρτηση αυτή επιτρέπει στον χρήστη να επιλέξει την επιθυμητή προβολή προς απεικόνιση, αντιστοιχίζοντας το πλήκτρο "1" στην ισομετρική προβολή, το πλήκτρο "2" στην προβολή \textlatin{Cavalier} και το πλήκτρο "3" στην προβολή \textlatin{Cabinet}. \par

Στο τελευταίο στάδιο δομείται η βασική συνάρτηση του προγράμματος, δηλαδή η \emph{\textlatin{main}} συνάρτηση, κατά την οποία ορίζεται ο τύπος και το παράθυρο απεικόνισης, δημιουργείται το προαναφερθέν παράθυρο και καλούνται οι συναρτήσεις απεικόνισης, ανασχηματισμού και επικοινωνίας με το πληκτρολόγιο, ώστε να εμφανιστεί στην οθόνη του χρήστη το παράθυρο των προβολών.

\newpage