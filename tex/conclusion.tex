% Συμπεράσματα

\addcontentsline{toc}{chapter}{Συμπεράσματα}
\chapter*{Συμπεράσματα}

Όπως έχει ήδη αναφερθεί, σκοπός της παρούσας εργασίας ήταν ο υπολογισμός των μηκών των πλαγίων προβολών \textlatin{Cavalier} και \textlatin{Cabinet} ενός μοναδιαίου κύβου στο επίπεδο $x-y$ με τη χρήση γωνίας αζιμουθίου, η υλοποίηση προγράμματος διαδραστικής περιστροφής του μοναδιαίου κύβου πέριξ των αξόνων $x$, $y$ και $z$, και η απεικόνιση των πλαγίων προβολών \textlatin{Cavalier} και \textlatin{Cabinet} και της προοπτικής προβολής σε τρία διαφορετικά παράθυρα. Όπως διαπιστώθηκε, ο υπολογισμός των μηκών των ακμών του κύβου, έχοντας εφαρμόσει πλάγια προβολή \textlatin{Cavalier} και \textlatin{Cabinet}, είναι εφικτός και δύναται να υλοποιηθεί με τον τρόπο που υποδείχθηκε στη θεωρητική υλοποίηση. Ως προς το δεύτερο ζητούμενο της εργασίας, συμπεράναμε πως η υλοποίηση στο λογισμικό \textlatin{OpenGL} ήταν εφικτή αλλά όχι και τόσο αβίαστη. \par

Δοθέντος αρκετού χρόνου για εμβάθυνση στο θέμα της εργασίας και εξοικείωση με την προαναφερθείσα προγραμματιστική πλατφόρμα, στοχεύουμε στην εξέταση και βελτιστοποίηση της υλοποίησης διαδραστικής περιστροφής μονοδιαίου κύβου και δημιουργίας πολλαπλών παραθύρων. Στους μελλοντικούς μας στόχους συμπεριλαμβάνεται και η μελέτη των υπολοίπων κατηγοριών προβολών αναφερθόντων επιγραμματικώς στο πλαίσιο της παρούσας εργασίας.
