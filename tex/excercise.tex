\chapter{Υπόδειξη μέτρησης μηκών ακμών σε προβολή \textlatin{Cavalier} και \textlatin{Cabinet}}

\counterwithout{equation}{chapter} % αφαιρεί αριθμό κεφαλαίου από τον αριθμό κάθε εξίσωσης

Στην ενότητα αυτή θα υλοποιηθεί η μέτρηση μηκών των προβολών των ακμών μοναδιαίου κύβου, οι οποίες αρχικώς ήταν κάθετες, και θα βρεθούν οι πλάγιες προβολές \textlatin{Cavalier} και \textlatin{Cabinet} του κύβου αυτού στο επίπεδο $x-y$ με τη χρήση γωνίας αζιμουθίου ίση με $60\degree$. Ως οδηγός για την επίλυση του προαναφερθέντος ζητήματος επιλέγεται παράδειγμα που προσδιορίζει το μητρώο πλαγίων προβολών καθορισμένων με γωνίες αζιμουθίου και ανύψωσης $φ$ και $θ$ αντιστοίχως, οι οποίες ορίζουν τη σχέση της κατεύθυνσης προβολής προς το επίπεδο προβολής (Δρακόπουλος, 2022).

\section{Πλάγια προβολή \textlatin{Cavalier}}
Αρχικώς, μελετούμε την πλάγια προβολή \textlatin{Cavalier}. Σε αυτή, απαιτείται γωνία ανύψωσης $θ = 45\degree$ και γωνία της επιλογής μας $φ = 60\degree$. Συγκεκριμένα, υποθέτουμε πως έχουμε μοναδιαίο κύβο με τον ακόλουθο πίνακα κορυφών.

\vspace{0.5em}

\begin{equation}
\textlatin{C} = \left(\begin{array}{rrrrrrrrr}
0 & 1 & 1 & 0 & 0 & 0 & 1 & 1\\
0 & 0 & 1 & 1 & 1 & 0 & 0 & 1\\
0 & 0 & 0 & 0 & 1 & 1 & 1 & 1\\
1 & 1 & 1 & 1 & 1 & 1 & 1 & 1\\
\end{array}\right)
\end{equation}\\
\\
Επίσης, υποθέτουμε πως $x-y$ είναι το επίπεδο στο οποίο προβάλλεται ο προαναφερθής μοναδιαίος κύβος. Το διάνυσμα κατεύθυνσης της προβολής είναι:\\
\begin{equation}
\overrightarrow{\textlatin{DOP}} = \left(\begin{array}{rrr}
\cos\theta \cos\phi, \cos\theta \sin\phi, \sin\theta
\end{array}\right)^{\mathrm T}
\end{equation}\\
όπου $φ$ είναι ο αριθμός αζιμουθίου και $θ$ η ανύψωση. Με αντικατάσταση των συνημιτόνων των γωνιών $φ$ και εφαπτομένων της γωνίας των  $60\degree$ και $θ$ της γωνίας ανύψωσης για την πλάγια προβολή \textlatin{Cavalier} στον παρακάτω πίνακα\\
\begin{equation}
\textlatin{Ρoblique}(\phi, \theta) = \left(\begin{array}{rrrr}
1 & 0 & -\cos\theta/\tan\theta & 0\\
0 & 1 & -\sin\phi/\tan\phi  & 1 \\
0 & 0 &     0 & 0 \\
0 & 0 &     0 & 1 \\
\end{array}\right)
\end{equation} \\
προκύπτει ο εξής πίνακας:\\
\begin{equation}
\textlatin{Ρoblique} (60, 45) = \left(\begin{array}{rrrr}
1 & 0 & -\surd2/2 & 0\\
0 & 1 & -1/2  & 1 \\
0 & 0 &  0 & 0 \\
0 & 0 & 0 & 1 \\
\end{array}\right)
\end{equation} \\

Στη συνέχεια, πραγματοποιείται πολλαπλασιασμός μεταξύ του \textlatin{POBLIQUE} και του πίνακα $C$ του μοναδιαίου κύβου και προκύπτει ο παράγων πίνακας. Ακολούθως, παρατηρούμε ποιες προβολές υπήρξαν κάθετες στο επίπεδο $x-y$ και λαμβάνουμε νόρμα μεταξύ των σημείων που ενώνονται σε κάθε προβολή. Εξετάζοντας τα αποτελέσματα των παραπάνω πράξεων, συμπεραίνουμε πως στις πλάγιες προβολές \textlatin{Cavalier} το αποτέλεσμα της νόρμας των δύο σημείων που ενώνονται σε κάθε προβολή είναι ίδιο με τον αρχικό αριθμό μήκους.

\section{Πλάγια προβολή \textlatin{Cabinet}}

Σε αυτό το σημείο, μελετάται η επίλυση της πλάγιας προβολής \textlatin{Cabinet}. Η προβολή \textlatin{Cabinet} απαιτεί γωνία ανύψωσης $θ = 63\degree$  και γωνία της επιλογής μας $φ = 60\degree$. Συγκεκριμένα, υποθέτουμε πως έχουμε τον ίδιο μοναδιαίο κύβο και το ίδιο επίπεδο προβολής $x-y$ με την προηγούμενη υλοποίηση. Το διάνυσμα κατεύθυνσης της προβολής και σε αυτή την περίπτωση είναι:\\
\begin{equation}
\overrightarrow{\textlatin{DOP}} = \left(\begin{array}{rrr}
\cos\theta\cos\phi, \cos\theta\sin\phi, \sin\theta
\end{array}\right)^{\mathrm T}
\end{equation} \\
όπου $φ$ είναι ο αριθμός αζιμουθίου και $θ$ η ανύψωση. Με αντικατάσταση των συνημίτονων των γωνιών $φ$ και εφαπτομένων της γωνίας των $60\degree$ και $θ$ της γωνίας ανύψωσης για την πλάγια προβολή \textlatin{Cabinet} στον παρακάτω πίνακα\\
\begin{equation}
\textlatin{Ρoblique}(\phi, \theta) = \left(\begin{array}{rrrr}
1 & 0 & -\cos\theta/\tan\theta & 0\\
0 & 1 & -\sin\phi/\tan\phi  & 1 \\
0 & 0 &     0 & 0 \\
0 & 0 &     0 & 1 \\
\end{array}\right)
\end{equation} \\
προκύπτει ο εξής πίνακας:\\
\begin{equation}
\textlatin{Ρoblique}(60, 63) = \left(\begin{array}{rrrr}
1 & 0 & -5 & 0\\
0 & 1 & -1/2  & 1 \\
0 & 0 &  0 & 0 \\
0 & 0 & 0 & 1 \\
\end{array}\right)
\end{equation} \\

Στη συνέχεια, πραγματοποιείται πολλαπλασιασμός μεταξύ του \textlatin{POBLIQUE} και του πίνακα $C$ του μοναδιαίου κύβου και προκύπτει ο παράγων πίνακας. Ακολούθως, παρατηρούμε ποιες προβολές υπήρξαν κάθετες στο επίπεδο $x-y$ και λαμβάνουμε νόρμα μεταξύ των σημείων που ενώνονται σε κάθε προβολή. Εξετάζοντας τα αποτελέσματα των παραπάνω πράξεων, συμπεραίνουμε πως στις πλάγιες προβολές \textlatin{Cabinet} το αποτέλεσμα της νόρμας των δύο σημείων που ενώνονται σε κάθε προβολή είναι μισό του αρχικού μήκους (Δρακόπουλος, 2022). \par
