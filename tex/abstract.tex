% Δεύτερη και τρίτη σελίδα: Περίληψη και λέξεις - κλειδιά (στην ελληνική και στην αγγλική)

\addcontentsline{toc}{chapter}{Περίληψη}
\chapter*{Περίληψη}

Η παρούσα εργασία εξετάζει τη δυνατότητα υπολογισμού των μηκών των ακμών των πλαγίων προβολών \textlatin{Cavalier} και \textlatin{Cabinet} ενός μοναδιαίου κύβου που προβάλλεται στο επίπεδο $x-y$ με τη χρήση γωνίας αζιμουθίου και την κατασκευή προγράμματος διαδραστικής περιστροφής κύβου και τύπωσης των προβολών  προοπτικής, \textlatin{Cavalier} και \textlatin{Cabinet} σε διαφορετικά παράθυρα. Αρχικά, εξοικειωνόμαστε με τα προς υλοποίηση ζητήματα και πραγματοποιούμε μία εισαγωγή στο γενικό θέμα των προβολών μοναδιαίου κύβου σε 3Δ σκηνή. Στη συνέχεια, μελετάμε το θεωρητικό υπόβαθρο της εργασίας, ορίζοντας και αναλύοντας την προβολή, τις υποκατηγορίες της και, συγκεκριμένα, τις πλάγιες προβολές \textlatin{Cavalier} και \textlatin{Cabinet}. Για την υλοποίηση του πρώτου ζητήματος, επιστρατεύουμε το δοθέν παράδειγμα υπολογισμού μητρώου πλάγιας προβολής με την χρήση γωνίας αζιμουθίου, με στόχο την εύρεση των μηκών των ακμών των πλαγίων προβολών \textlatin{Cavalier} και \textlatin{Cabinet} του μοναδιαίου κύβου με την χρήση γωνίας αζιμουθίου. Για την υλοποίηση του δευτέρου ζητήματος, αναγκαία είναι η χρήση της προγραμματιστικής γλώσσας \textlatin{C++} και του λογισμικού \textlatin{OpenGL}. Βάσει των συμπερασμάτων, διαπιστώνουμε πως ευκόλως υλοποιείται η διαδικασία υπολογισμού μηκών ακμών σε κύβο πλαγίων προβολών \textlatin{Cavalier} και \textlatin{Cabinet}, καθώς και η κατασκευή διαδραστικού περιστρεφόμενου κύβου με παράθυρο προοπτικής προβολής, προβολής \textlatin{Cavalier} και προβολής \textlatin{Cabinet}. Τέλος, θέτουμε μελλοντικούς στόχους μελέτης και εφαρμογής των υπολοίπων κατηγοριών προβολών που αναφέρθηκαν επιγραμματικά εντός του πλαισίου της εργασίας, καθώς και υλοποίησής τους με τη χρήση του λογισμικού \textlatin{OpenGL}.

\vspace{1.5em}

\section*{Λέξεις - κλειδιά}
Πλάγια προβολή, \textlatin{Cavalier}, \textlatin{Cabinet}, μοναδιαίος κύβος, γωνία αζιμουθίου 

\newpage

\chapter*{\textlatin{Abstract}}

\textlatin{The present study examines the possibility of calculating the lengths of the edges of the Cavalier and Cabinet oblique projections of a unit cube projected in an $x-y$ plane using the azimuth angle and developing a program for the interactive rotation of a cube and the printing of perspective projection, Cavalier projection, and Cabinet projection in different windows. First, we get acquainted with the issues to be implemented, and make an introduction about the general subject of unit cube projections in the 3D scene. Next, we study the theoretical background of the project, defining and analyzing projection, its subcategories and, specifically, the Cavalier and Cabinet oblique projections. For the solution of the first problem, we employ the given example of matrix calculation for the Cavalier and Cabinet oblique projections of the unit cube using the azimuth angle. For the implementation of the second issue, it is necessary to use the C++ programming language and the OpenGL software. Based on the conclusions, we realize that the process of calculating the lengths of the edges of a unit cube of the Cavalier and Cabinet oblique projections, as well as the development of an interactive rotating cube with a perspective projection window, a Cavalier projection window, and a Cabinet projection window, are easily implemented. Finally, we set future goals for the study and implementation of the rest of the projection categories mentioned briefly within the context of this work, as well as their application using the OpenGL software.}

\vspace{1.5em}

\section*{\textlatin{Key Words}}
\textlatin{Oblique projection, Cavalier, Cabinet, unit cube, azimuth angle}
